\documentclass[]{article}

\usepackage{graphicx}
\graphicspath{ {Images/} }
\usepackage{titlesec}
\usepackage{wrapfig}

\titleformat{\subsection}
  {\normalfont\HUGE\bfseries}{\thesection}{1em}{}[{\titlerule[10.0pt]}]

\titleformat{\subsection}
  {\normalfont\Large\bfseries}{\thesection}{1em}{}[{\titlerule[0.8pt]}]

\begin{document}

\section*{IACS Computes! 2017 Teaching Assistants}

\subsection*{Joel Anderson} 
\begin{wrapfigure}[6]{R}{0.2\textwidth}
\begin{centering}
    \includegraphics[width=0.2\textwidth]{joel.jpg}
\end{centering}
\end{wrapfigure}
I'm a student in the PhD program in the Applied Mathematics and Statistics department here at Stony Brook University. I use programming every day in my research into numerical methods for relativistic quantum chemistry. I took my first programming course (in Python!) my senior year in high school, and now I primarily program in C++.

\subsection*{Marisa Lim} 
\begin{wrapfigure}[6]{R}{0.2\textwidth}
%\begin{centering}
%    \includegraphics[width=0.2\textwidth]{marisa.jpg}
%\end{centering}
\end{wrapfigure}



\subsection*{Bryan Sundahl} 
\begin{wrapfigure}[4]{R}{0.2\textwidth}
\begin{centering}
    \includegraphics[width=0.2\textwidth]{bryan.jpg}
\end{centering}
\end{wrapfigure}
I'm a graduate student in Applied Math and Statistics here at Stony Brook University, working in the lab of Robert Harrison. My research involves creating software to model the response of molecules to external perturbations (\textit{e.g.} lasers).  Python is the main tool in my analysis of these models. 

\vspace{0.5 in}

\subsection*{Rebecca Uliasz} 
\begin{wrapfigure}{R}{0.2\textwidth}
%\begin{centering}
%    \includegraphics[width=0.193\textwidth]{rebecca.jpg}
%\end{centering}
\end{wrapfigure}


\end{document}
